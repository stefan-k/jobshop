\section{Open Job Shop Problem}

%%%%%%%%%%%%%%%%%%%%%%%%%%%%%%%%%%%%%%%%%%%%%%%%%%%%%%%%%%%%%%%%%%%%%%%%%%%%%%%%
\subsection{Introduction}

%%%%%%%%%%%%%%%%%%%%%%%%%%%%%%%%%%%%%%%%%%%%%%%%%%%%%%%%%%%%%%%%%%%%%%%%%%%%%%%%


\begin{frame}
  \frametitle{Open Job Shop Scheduling Problem}

  \begin{itemize}
    \item $m$ \textbf{jobs} have to processed by $n$ \textbf{machines}
    \item A job $i$ stays on a machine $j$ for an \textbf{operation} with start time $t_{ij}$ and duration $d_{ij}$
    \item The order in which a job is passed from machine to machine can be chosen arbitrarily.
  \end{itemize}

  Goal: \textbf{minimize} the total \textbf{makespan}, i.e. the time at which the last machine finishes its last operation.

 	

\end{frame}

%%%%%%%%%%%%%%%%%%%%%%%%%%%%%%%%%%%%%%%%%%%%%%%%%%%%%%%%%%%%%%%%%%%%%%%%%%%%%%%%

\begin{frame}
	\frametitle{Example}
	
	\begin{columns}[c]
		\column{.4\linewidth}

Valid schedule for a 3x3-problem (3 jobs, 3 machines)

\textbf{makespan:} 23, matches lower bound

		\column{.6\linewidth}
\begin{figure}
	\includegraphics[width=\linewidth]{images/example-schedule.pdf}
	
\end{figure}
\end{columns}
	
\end{frame}

%%%%%%%%%%%%%%%%%%%%%%%%%%%%%%%%%%%%%%%%%%%%%%%%%%%%%%%%%%%%%%%%%%%%%%%%%%%%%%%%

\begin{frame}
	\frametitle{$NP$-hardness}
	
	\begin{itemize}
	
		\item Similar to normal Job Shop and Flow Shop problems, but with a larger solution space \textrightarrow \hspace{.5em} harder to find optimal solution.

		\item 	The Open Job Shop Scheduling Problem is known to be $NP$-hard for $n \geq 3$ (the Partitioning Problem can be reduced to it).

		\item 	\textrightarrow \hspace{.5em} Research focusses on \textbf{heuristics} since exact approaches are not feasible.
	\end{itemize}
	
	
\end{frame}

%%%%%%%%%%%%%%%%%%%%%%%%%%%%%%%%%%%%%%%%%%%%%%%%%%%%%%%%%%%%%%%%%%%%%%%%%%%%%%%%

\begin{frame}
	\frametitle{Known Approaches}
	\begin{itemize}
	
		\item Branch and Bound (1997)

		\item 	Tabu Search (1999)

		\item 	\textbf{Evolutionary based heuristics} (1999)

		\item 	Particle Swarm Optimization (2008)
	\end{itemize}
	
	
\end{frame}

%%%%%%%%%%%%%%%%%%%%%%%%%%%%%%%%%%%%%%%%%%%%%%%%%%%%%%%%%%%%%%%%%%%%%%%%%%%%%%%%

\subsection{Algorithms}

%%%%%%%%%%%%%%%%%%%%%%%%%%%%%%%%%%%%%%%%%%%%%%%%%%%%%%%%%%%%%%%%%%%%%%%%%%%%%%%%

\begin{frame}
  \frametitle{Permutation Genetic Algorithm\cite{Khuri}}
	
	\begin{idea}
	
	
	\begin{itemize}
	
		\item Give each operation a unique $ID = 1,2,\dots, m \cdot n$

		\item 	Let the order of the IDs decide which operation is \textbf{scheduled} first.

		\item 	Use the \emph{genetic algorithm} to find the best \emph{permutation}.
	\end{itemize}
	
	\end{idea}
	
	\pause
	
	What does it mean to \emph{schedule} an operation?  

\end{frame}

%%%%%%%%%%%%%%%%%%%%%%%%%%%%%%%%%%%%%%%%%%%%%%%%%%%%%%%%%%%%%%%%%%%%%%%%%%%%%%%%

\begin{frame}
  \frametitle{Scheduling step}
	
	
	\small
	\begin{columns}	
	\column{.5\textwidth}
Simple scheduling: Append operation to machine, after previous operations of the same job have finished.



\column{.5\linewidth}
	\begin{figure}
		\includegraphics[width=.8\linewidth]{images/scheduling.png}

	\end{figure}
	
\end{columns}

\textbf{Better:} Use \textbf{gaps} (no other operation of the same job should be active)!

	A significant part of the optimization is ``hidden'' in this scheduling step!

\end{frame}

%%%%%%%%%%%%%%%%%%%%%%%%%%%%%%%%%%%%%%%%%%%%%%%%%%%%%%%%%%%%%%%%%%%%%%%%%%%%%%%%

\begin{frame}
    \frametitle{Implementation details}
    \begin{itemize}

    	\item A chromosome consists of $m\cdot~n$ genes, each gene representing an operation.

    	\item Start with random genes

    	\item Genetic algorithm adapts genes freely as floats \textrightarrow \hspace{.5em} the order of the genes decides the permutation, e.g.

    	\item a chromosome $(64.4, 33.5, 64.5)$ represents the permutation $(2,1,3)$
    \end{itemize}
\end{frame}

%%%%%%%%%%%%%%%%%%%%%%%%%%%%%%%%%%%%%%%%%%%%%%%%%%%%%%%%%%%%%%%%%%%%%%%%%%%%%%%%

\begin{frame}
  \frametitle{Hybrid Genetic Algorithm}
\begin{itemize}

	\item 	Uses the same genetic algorithm with the same parameters, but

	\item  	permutes jobs instead of operations

	\item 	Called \textbf{hybrid}, because it keeps a sorted list of unfinished operations for each job \textrightarrow \hspace{.5em} schedules long operations first
	
\end{itemize}


 
\end{frame}


%%%%%%%%%%%%%%%%%%%%%%%%%%%%%%%%%%%%%%%%%%%%%%%%%%%%%%%%%%%%%%%%%%%%%%%%%%%%%%%%

\begin{frame}
  \frametitle{Hybrid Genetic - Example}

\begin{figure}[htbp]
	\centering
		\includegraphics[scale=1]{images/hyb0.png}
	\caption{The table represents the problem definition, the sequence ``1 3 2 4'' directs the scheduling order.}
	\label{fig:label}
\end{figure}
	
 
\end{frame}
\begin{frame}
  \frametitle{Hybrid Genetic - Example (1)}

\begin{figure}[htbp]
	\centering
		\includegraphics[scale=1]{images/hyb1.png}
	\caption{First, schedule the longest operation of job 1 (left column).}
	\label{fig:label}
\end{figure}
	
 
\end{frame}
\begin{frame}
  \frametitle{Hybrid Genetic - Example (2)}

\begin{figure}[htbp]
	\centering
		\includegraphics[scale=1]{images/hyb2.png}
	\caption{The index 3 also points to job 1 (modulo). Schedule the second longest operation.}
	\label{fig:label}
\end{figure}
	
 
\end{frame}
\begin{frame}
  \frametitle{Hybrid Genetic - Example (3)}

\begin{figure}[htbp]
	\centering
		\includegraphics[scale=1]{images/hyb3.png}
	\caption{Now, any index would refer to job 2, because it's the only unfinished job. Schedule the longest operation.}
	\label{fig:label}
\end{figure}
	
 
\end{frame}
\begin{frame}
  \frametitle{Hybrid Genetic - Example (4)}

\begin{figure}[htbp]
	\centering
		\includegraphics[scale=1]{images/hyb4.png}
	\caption{Finish scheduling.}
	\label{fig:label}
\end{figure}
	
 
\end{frame}

%%%%%%%%%%%%%%%%%%%%%%%%%%%%%%%%%%%%%%%%%%%%%%%%%%%%%%%%%%%%%%%%%%%%%%%%%%%%%%%%

\begin{frame}
  \frametitle{Selfish Gene Algorithm}

\begin{itemize}

	\item Uses a \textbf{virtual population} (VP)%, does \textbf{not} use the evolib!

	\begin{itemize}
		\item instead of storing chromosomes, store probabilities of gene values.

		\item Indivuals are drawn from the VP according to these probabilities.

		\item Better fitness means higher probability for genes (a competition is held).
	
		\item \textbf{steady state}: Every gene acquires a certain value with  $P \geq 95\%$.
	\end{itemize}
	\item In our case: genes represent job/machine indices.
\end{itemize}

\end{frame}

%%%%%%%%%%%%%%%%%%%%%%%%%%%%%%%%%%%%%%%%%%%%%%%%%%%%%%%%%%%%%%%%%%%%%%%%%%%%%%%%

\begin{frame}[fragile]
	\frametitle{Selfish Gene Algorithm - Pseudocode}
	
\vspace{-0.8cm}
{\scriptsize
\begin{verbatim}
SELFISH_GENE():
    
    population = [[1/m,1/n,...],[1/m,1/n,...], ...]
    best = choose individual from population
    
    FOR i = 1:max_iterations
        choose individual1, individual2 from population
        (winner, loser) = compare(individual1, individual2)
        reward(winner)
        punish(loser)
        IF fitness(winner) > fitness(best)
            best = winner
        END
        BREAK if steady_state(population)
    END
    
    RETURN best
\end{verbatim}}

\end{frame}

%%%%%%%%%%%%%%%%%%%%%%%%%%%%%%%%%%%%%%%%%%%%%%%%%%%%%%%%%%%%%%%%%%%%%%%%%%%%%%%%
